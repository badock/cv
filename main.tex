\documentclass[french,11pt,a4paper]{moderncv}

\usepackage[frenchb]{babel}
\usepackage[utf8]{inputenc}  
\usepackage[T1]{fontenc}
\RequirePackage{filecontents}


%% \usepackage{apacite}
\usepackage[backend=biber,sorting=ydnt]{biblatex}
\addbibresource{main.bib}

\usepackage[scale=0.75]{geometry}

%% \usepackage{multibib}
%% \newcites{article,book}{{Articles},{Books}}
%% \newcites{journal}{Journal References}

\moderncvstyle{casual}
\moderncvcolor{green}
\usepackage[scale=0.75]{geometry}
\name{Jonathan}{Pastor}

\nopagenumbers

%\phone[fixed]{}
%\email{}
%\homepage{}
%\social[github]{}


%----------------------------------------------------------------------------------
%            content
%----------------------------------------------------------------------------------
\begin{document}
\makecvtitle

\vspace{-1.3cm}

\textbf{Jonathan Pastor}

\vspace{5mm} %5mm vertical space

%% \textbf{Adresse:}
%% \\
%% 15 Rue des soupirs, 44300 Nantes, FRANCE

%% \textbf{Tél. portable:} 06 58 21 09 06

\textbf{E-mail:} jonathan.pastor@me.com

\textbf{Born on:} 05/09/1987 (DD/MM/YYYY)

\textbf{Citizenship:} French

%% \section{Présentation générale}

%% Je suis post-doctorant au sein de l'équipe STACK au laboratoire des
%% Sciences du Numérique de Nantes (LS2N), où je travaille sur la mise en
%% place de la plateforme SeDuCe, intégrée avec Grid'5000, qui permet
%% l'étude des aspects énergétiques des centres de données. J'ai soutenu
%% ma thèse de doctorat le 18 octobre 2016, qui portait sur la conception
%% d'une infrastructure de Cloud Computing massivement
%% distribuée. Pendant ma thèse, j'ai effectué des activités
%% d'enseignement à l'École des Mines de Nantes. En parallèle de
%% l'écriture de mon manuscrit de thèse, j'ai travaillé pendant un an à
%% l'université de Chicago en qualité d'ingénieur de Recherche sur la
%% plateforme ``Chameleon''.

%% \section{Thèse de doctorat}
%% \cvitem{title}{\emph{Contributions à la mise en place d'une infrastructure de Cloud Computing à large échelle.}}
%% \cvitem{supervisors}{Frédéric Desprez et Adrien Lebre}
%% \cvitem{defense}{Octobre 2016}

%% \section{Education}

%% \cventry{2009--2012}{Master of Science, Computer Science}{Ecole des Mines de Nantes}{Nantes}{\textit{Engineering degree}}{}
%% \cventry{2009--2012}{Bachelor, Computer Science}{Université de Nantes}{Nantes}{\textit{}}{}


\section{Education and professional experiences}

\cventry{Since september 2020}{R\&D engineer}{Easyvirt}{Nantes}{}{Member of Research and development team. Setting up and extension of the continuous integration platform. Contributions to the \textit{DCscope} software. Development of a module dedicated to the collect and analysis of network communications using the \textit{Netflow} protocol.}

\cventry{2017--2020}{Post-doctoral researcher}{STACK team - IMT Atlantique}{Nantes}{}{Setting up of the SeDuCe platform, integrated with Grid'5000, which enables to study energetic aspects (electric consumption, thermal aspects and usage of renewable energies using solar panels) of datacenters. Engineer on the Grid'5000 platform in charge of managing clusters located in Nantes, collaborating with the technical committee on software development.}

\cventry{2016}{Research engineer}{Nimbus team}{University of Chicago/Argonne National Laboratory}{}{Participation to the setting up of the ``Chameleon'' testbed, and academic Cloud Computing infrastructure, and development of a software that enables the deployment of software clusters on academic Cloud Computing infrastructures based on OpenStack.}

\vspace{-0.02cm} % Hack to ensure that the following item is on the first page

\cventry{2012--2016}{Ph.D thesis (computer science)}{ASCOLA team - École
  des Mines de Nantes}{Nantes}{\textit{}}{\textit{Titre: Contributions to the setting up of a large-scale Cloud Computing infrastructure.}
  \vspace{0.5mm}
   Defended on octobre 18th 2016 with the following jury:
  \begin{itemize}
  \item M. \textcolor{blue}{Mario SÜDHOLT}, Professor, École des Mines de Nantes, \textit{Président du jury};
  \item M. \textcolor{blue}{Pierre SENS}, Professor, LIP6, \textit{Rapporteur};
  \item M. \textcolor{blue}{Stéphane GENAUD}, Professor, ENSIIE, \textit{Rapporteur};
  \item M. \textcolor{blue}{Thierry COUPAYE}, Directeur de domaine de recherche, Orange Labs, \textit{Examinateur};
  \item M. \textcolor{blue}{Frédéric DESPREZ}, Directeur de recherche, Université de Grenoble, \textit{Directeur};
  \item M. \textcolor{blue}{Adrien LEBRE}, Researcher, INRIA, \textit{Co-directeur};    
  \end{itemize}
  }

\cventry{2012}{Research internship}{ASCOLA team - École des Mines de Nantes}{Nantes}{}{Participation to the development of chemical programming system dedicated to distributed systems.}

\cventry{2009--2012}{Engineer degree (Master of science)}{Ecole des Mines de Nantes}{Nantes}{\textit{Génie des systèmes informatiques}}{}

\cventry{2006--2009}{Bachelor degree}{Université de Nantes}{Nantes}{Spécialité \textit{Computer science}}{}

\section{Teaching}

\cventry{2017--2020}{Teaching assistant in Computer Science Department (DAPI)}{IMT Atlantique}{Nantes}{}{Website development based on Python and Javascript (lectures and practical sessions, 2017-2020), supervision of several students projects (2017, 2019)}

\cventry{2012--2015}{Teaching assistant in Computer Science Department}{École des Mines de Nantes}{Nantes}{}{Website development based on Python and Javascript (lectures and practical sessions, 2012-2015), databases (practical sessions, 2012), object programming (exercices sessions and practical sessions, 2014), data structures (practical sessions, 2014), Scala (practical sessions, 2013-2014), Javascript (practical sessions, 2013), supervision of several students projects (2013-2015)}


\section{Skills}
\cvitem{Programmation}{Python, Bash, C/C++/Arduino, Scala, Java, Arduino}
\cvitem{Programmation Web}{HTML, CSS, Javascript, Bootstrap, Flask, Angular, VueJS}
\cvitem{Systèmes}{Linux, OpenStack, VMware vSphere, Grid'5000, UNIX, MariaDB/MySQL, Redis, RaspberryPI, MacOS}

\cvitem{English}{Fluent}

%% \section{Activités de recherches}

\section{Awards}

\cventry{September 2018}{Best paper award}{}{IARA GREEN 2018}{Venise,
  Italie}{SeDuCe: a Testbed for Research on Thermal and Power
  Management in Datacenters.}

\cventry{Juin 2014}{Grid'5000 large-scale challenge}{1er prix}{Grid'5000 Spring school 2014}
        {Ecole normale supérieure de Lyon}{The experiment I designed and conducted, in collaboration with  Laurent Pouilloux, won the first prize of the ``large-scale'' challenge.
          With this experiment, we demonstrated the scalability of DVMS combined with the Vivaldi algorithm to
  deploy and schedule dynamically 1700 virtual machines with a dynamic workload on a multi-site infrastructure.}


\subsection{Research Interests}
\cvitem{Cloud Computing}{Cloud Computing and Fog/Edge Computing. Infrastructure As A Service (OpenStack, Kubernetes)}
\cvitem{Energetic aspects in datacenters}{electrical consumption, thermal dissipation, renewable energies}
\cvitem{Distributed systems}{geo-distribued systems, large-scale infrastructures, fault tolerance}

\section{Publications}
\nocite{*}

%% %% \bibliographystyle{ieeetr}
%% %% \bibliography{main}

\printbibliography[prefixnumbers={A},type=incollection,title={Journals / Book chapters},heading=subbibliography]

\printbibliography[prefixnumbers={A},type=article,title={Articles},heading=subbibliography]


\printbibliography[prefixnumbers={P},type=inproceedings,title={International conferences},
                   heading=subbibliography, notkeyword={confnat}]
                   

\printbibliography[prefixnumbers={P},type=inproceedings,title={French conferences},
                   heading=subbibliography, keyword={confnat}]

\printbibliography[prefixnumbers={R},type=report,title={Reports},heading=subbibliography]

%% \section{Presentations}

\end{document}
