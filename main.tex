\documentclass[french,11pt,a4paper,sans]{moderncv}

\usepackage[frenchb]{babel}
\usepackage[utf8]{inputenc}  
\usepackage[T1]{fontenc}
\RequirePackage{filecontents}


%% \usepackage{apacite}
\usepackage[backend=biber,sorting=ydnt]{biblatex}
\addbibresource{main.bib}

\usepackage[scale=0.75]{geometry}

%% \usepackage{multibib}
%% \newcites{article,book}{{Articles},{Books}}
%% \newcites{journal}{Journal References}

\moderncvstyle{casual}
\moderncvcolor{green}
\usepackage[scale=0.75]{geometry}
\name{}{}

\nopagenumbers

%\phone[fixed]{}
%\email{}
%\homepage{}
%\social[github]{}


%----------------------------------------------------------------------------------
%            content
%----------------------------------------------------------------------------------
\begin{document}
\makecvtitle

\textbf{Jonathan Pastor}

\vspace{5mm} %5mm vertical space

\textbf{Adresse professionnelle:}
\\
IMT Atlantique, 4 Rue Alfred Kastler, 44300 Nantes FRANCE

\textbf{tél. portable:} 06 58 21 09 06

\textbf{e-mail:} jonathan.pastor@imt-atlantique.fr

\textbf{Né le:} 05/09/1987

\textbf{nationalité:} Française

\section{Présentation générale}

Je suis postdoctorant au sein de l'équipe STACK au sein du Laboratoire
des Sciences du Numérique de Nantes (LS2N), où je travaille sur la
mise en place de la plateforme SeDuCe, intégrée avec Grid'5000, qui
permet l'étude des aspects énergétiques des centres de données. J'ai
soutenu ma thèse de doctorat le 18 octobre 2016, en soutenant une
thèse portant sur la conception d'une infrastructure de Cloud
Computing massivement distribuée. Pendant ma thèse j'ai effectué des
activités d'enseignement à l'École des Mines de Nantes. En parallèle
de l'écriture de mon manuscrit de thèse, j'ai travaillé pendant un an
à l'université de Chicago en qualité d'ingénieur de Recherche sur la
plateforme ``Chameleon''.

%% \section{Thèse de doctorat}
%% \cvitem{title}{\emph{Contributions à la mise en place d'une infrastructure de Cloud Computing à large échelle.}}
%% \cvitem{supervisors}{Frédéric Desprez et Adrien Lebre}
%% \cvitem{defense}{Octobre 2016}

%% \section{Education}

%% \cventry{2009--2012}{Master of Science, Computer Science}{Ecole des Mines de Nantes}{Nantes}{\textit{Engineering degree}}{}
%% \cventry{2009--2012}{Bachelor, Computer Science}{Université de Nantes}{Nantes}{\textit{}}{}


\section{Formation et expériences}

\cventry{2017--2020}{Postdoctorat}{équipe STACK - IMT Atlantique}{Nantes}{}{Mise en place de la plateforme SeDuCe, intégrée à Grid'5000, permettant l'étude énergétique (consommation éléctrique, thermique et approvisionnement en énergies renouvelbles). J'assurait en même temps une charge d'ingénieur Grid'5000, m'occupant des clusters de Nantes et aidant sur des développements logiciels.}

\cventry{2016}{Ingénieur de recherche}{Équipe Nimbus}{Université de Chicago/Argonne National Laboratory}{}{Travail sur la mise en place de ``Chameleon'', une infrastructure de Cloud Computing académique, et développement d'un logiciel permettant de mettre en place des clusters logiciels élastiques sur des infrastructures de Cloud Computing académiques basées sur OpenStack.}

\cventry{2012--2016}{Doctorat en informatique}{équipe ASCOLA - École
  des Mines de Nantes}{Nantes}{\textit{}}{\textit{titre:Contributions à la mise en place d'une infrastructure de Cloud Computing à large échelle.}
  \vspace{0.5mm}
   Soutenue le 18 octobre 2016 devant un jury composé de:
  \begin{itemize}
  \item M. \textcolor{blue}{Mario SÜDHOLT}, Professeur, École des Mines de Nantes, \textit{président du jury}.
  \item M. \textcolor{blue}{Pierre SENS}, Professeur des Universités, LIP6, \textit{rapporteur}.
  \item M. \textcolor{blue}{Stéphane GENAUD}, Professeur des Universités, ENSIIE, \textit{rapporteur}.
  \item M. \textcolor{blue}{Thierry COUPAYE}, Directeur de domaine de recherhche, Orange Labs, \textit{examinateur}.
  \item M. \textcolor{blue}{Frédéric DESPREZ}, Directeur de recherche, Université de Grenoble, \textit{directeur}.
  \item M. \textcolor{blue}{Adrien LEBRE}, Chargé de recherche, INRIA, \textit{co-directeur}.    
  \end{itemize}
  }

\cventry{2012}{Stage de recherche}{équipe ASCOLA - École des Mines de Nantes}{Nantes}{}{Participation au développement d'un système de programmation chimique.}

\cventry{2009--2012}{Diplôme d'ingénieur (master)}{Ecole des Mines de Nantes}{Nantes}{Spécialité \textit{Génie des systèmes informatiques}}{}

\cventry{2006--2009}{Licence d'informatique}{Université de Nantes}{Nantes}{Spécialité \textit{informatique}}{}

\section{Enseignements}

\cventry{2017--2020}{Vacataire au département informatique et productique (DAPI)}{IMT Atlantique}{Nantes}{}{Création de sites web (CMs et TPs, 2017-2020), encadrements de projets étudiants (2017, 2019)}

\cventry{2012--2015}{Vacataire au département informatique}{École des Mines de Nantes}{Nantes}{}{Création de sites web (CMs et TPs, 2012-2015), Bases de données (TPs, 2012), Programmation modulaire (TDs et TPs, 2014), Structures algorithmiques (TPs, 2014), Scala (TPs, 2013-2014), Javascript (TPs, 2013), encadrements de projets étudiants (2013-2015)}


\section{Compétences}
\cvitem{Programmation}{Python, Bash, C/C++/Arduino, Scala/Akka, Java, Arduino, Ruby}
\cvitem{Programmation Web}{HTML, CSS, Javascript, Bootstrap, Flask, VueJS}
\cvitem{Systèmes}{Linux, OpenStack, Grid'5000, UNIX, MariaDB/MySQL, Redis, RaspberryPI, MacOS}

\cvitem{Anglais}{Bon niveau (lu, écrit, parlé)}

%% \section{Activités de recherches}

\section{Récompenses}

\cventry{September 2018}{Prix du meilleur papier}{}{GREEN 2018}{Venise,
  Italie}{SeDuCe: a Testbed for Research on Thermal and Power
  Management in Datacenters.}

\cventry{Juin 2014}{Grid'5000 large scale challenge}{1er prix}{École
  d'été Grid'5000 2014}{Ecole normale supérieure de Lyon}{L'expérience
  que j'ai conduite avec Laurent Pouilloux a remporté le premier prix
  du défi ``large échelle'' (large scale challenge). Au cours de cette
  expérience, nous avons utilisé DVMS et l'algorithme Vivaldi pour
  déployer et ordonnancer 1700 machines virtuelles sur une
  infrastructure multi-site.}


%% \subsection{Research Interests}
%% \cvitem{Cloud Computing}{Geographically distributed  OpenStack, Cloud Computing infrastructures.}
%% \cvitem{Distributed computing}{Internet of things, Distributed algorithms, concurrency, fault-tolerance.}

%% \section{Publications}
%% \nocite{*}

%% %% \bibliographystyle{ieeetr}
%% %% \bibliography{main}

%% \printbibliography[prefixnumbers={A},type=incollection,title={Journal / Book chapter},heading=subbibliography]

%% \printbibliography[prefixnumbers={A},type=article,title={Articles},heading=subbibliography]


%% \printbibliography[prefixnumbers={P},type=inproceedings,title={International conferences},
%%                    heading=subbibliography, notkeyword={confnat}]
                   

%% \printbibliography[prefixnumbers={P},type=inproceedings,title={French conferences},
%%                    heading=subbibliography, keyword={confnat}]

%% \printbibliography[prefixnumbers={R},type=report,title={Reports},heading=subbibliography]

%% \section{Presentations}

\end{document}
